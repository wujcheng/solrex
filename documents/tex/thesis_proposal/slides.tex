\documentclass[mathserif,compress,CJKutf8, red]{beamer}

\mode<presentation>
{
  \usetheme{Antibes}
  % 可供选择的主题参见 beameruserguide.pdf, 第 134 页起
  % 无导航条的主题: Bergen, Boadilla, Madrid, Pittsburgh, Rochester;
  % 有树形导航条的主题: Antibes, JuanLesPins, Montpellier;
  % 有目录竖条的主题: Berkeley, PaloAlto, Goettingen, Marburg, Hannover;
  % 有圆点导航条的主题: Berlin, Dresden, Darmstadt, Frankfurt, Singapore, Szeged;
  % 有节与小节导航条的主题: Copenhagen, Luebeck, Malmos, Warsaw

% \setbeamercovered{transparent}
% 如果取消上一行的注解 %, 就会使得被覆盖部分变得透明(依稀可见)
}

\usepackage{CJKutf8}   % 文档使用 UTF8 编码
\usepackage{hyperref}
\usepackage{amsmath,amssymb,amsfonts}
\usepackage{color,xcolor}
\usepackage{graphicx}
\usepackage{manfnt}

\hypersetup{bookmarks=true, % 是否使用书签
                   unicode, % 使用 unicode 编码书签
          colorlinks=false, % 是否使用彩色链接
         pdfborder={0 0 0}, % 链接周围有无边框,{0 0 0}表示无
        pdfstartview=FitBH, % 
    pdfpagemode=FullScreen, % 全屏显示
    pdfauthor={Wenbo Yang}, % pdf 作者
   pdftitle={WSN Security}, % pdf 标题
 pdfsubject={WSN Security}, % pdf 主题
}

%%%%%%%%%%%%%%%%%%%%%%%%%%%%%%%%%%%%%%%%%%%%%%%%%%%%%%%%%%%%%%%%%%%%%%%%%%%%
%                          定制幻灯片---重定义字体、字号命令                           %
%%%%%%%%%%%%%%%%%%%%%%%%%%%%%%%%%%%%%%%%%%%%%%%%%%%%%%%%%%%%%%%%%%%%%%%%%%%%
\newcommand{\song}{\CJKfamily{song}}    % 宋体   (Windows自带simsun.ttf)
\newcommand{\fs}{\CJKfamily{fs}}        % 仿宋体 (Windows自带simfs.ttf)
\newcommand{\kai}{\CJKfamily{kai}}      % 楷体   (Windows自带simkai.ttf)
\newcommand{\hei}{\CJKfamily{hei}}      % 黑体   (Windows自带simhei.ttf)
\newcommand{\li}{\CJKfamily{li}}        % 隶书   (Windows自带simli.ttf)
\newcommand{\you}{\CJKfamily{you}}      % 幼圆   (Windows自带simyou.ttf)
\newcommand{\chuhao}{\fontsize{42pt}{\baselineskip}\selectfont}     % 字号设置
\newcommand{\xiaochuhao}{\fontsize{36pt}{\baselineskip}\selectfont} % 字号设置
\newcommand{\yichu}{\fontsize{32pt}{\baselineskip}\selectfont}      % 字号设置
\newcommand{\yihao}{\fontsize{28pt}{\baselineskip}\selectfont}      % 字号设置
\newcommand{\erhao}{\fontsize{21pt}{\baselineskip}\selectfont}      % 字号设置
\newcommand{\xiaoerhao}{\fontsize{18pt}{\baselineskip}\selectfont}  % 字号设置
\newcommand{\sanhao}{\fontsize{15.75pt}{\baselineskip}\selectfont}  % 字号设置
\newcommand{\sihao}{\fontsize{14pt}{\baselineskip}\selectfont}      % 字号设置
\newcommand{\xiaosihao}{\fontsize{12pt}{\baselineskip}\selectfont}  % 字号设置
\newcommand{\wuhao}{\fontsize{10.5pt}{\baselineskip}\selectfont}    % 字号设置
\newcommand{\xiaowuhao}{\fontsize{9pt}{\baselineskip}\selectfont}   % 字号设置
\newcommand{\liuhao}{\fontsize{7.875pt}{\baselineskip}\selectfont}  % 字号设置
\newcommand{\qihao}{\fontsize{5.25pt}{\baselineskip}\selectfont}    % 字号设置

\newcommand{\CJKtoday}{\number\year 年 \number\month 月 \number\day 日}

\begin{document}
\begin{CJK*}{UTF8}{song}
\CJKtilde
%======================= 标题名称中文化 ============================%
\newtheorem{dingyi}{\hei 定义~}[section]
\newtheorem{dingli}{\hei 定理~}[section]
\newtheorem{yinli}[dingli]{\hei 引理~}
\newtheorem{tuilun}[dingli]{\hei 推论~}
\newtheorem{mingti}[dingli]{\hei 命题~}

\makeatletter
\renewcommand\theequation{\thesection.\arabic{equation}}
\@addtoreset{equation}{section}
\makeatother

\title[~WSN~中的位置相关安全问题研究]{无线传感器网络中的位置相关安全问题研究}   % 如果标题不长, [短标题]可以略去
%\subtitle{副标题}

\author[杨文博]%
{\hei 杨文博\\~\\电子邮件:~\href{mailto:wbyang@is.ac.cn}{\textcolor{blue}{\texttt{wbyang@is.ac.cn}}}}

\institute{\sihao\kai 信息安全国家重点实验室}

\date{\CJKtoday}

\titlegraphic{\includegraphics[height=0.17\textwidth]{images/lois.png}} % 徽章

\begin{frame}
  \titlepage
\end{frame}

\begin{frame}{大纲}
  \tableofcontents
\end{frame}

% 除掉以下命令的注解 "%" 后, 许多环境都会自动逐段显示
%\beamerdefaultoverlayspecification{<+->}

\section{背景}

\begin{frame}{无线传感器网络(WSN)}

未完成\cite{Lee2007}
\end{frame}

\section{安全定位}

\begin{frame}{WSN~中的安全定位问题}

未完成
\end{frame}

\section{位置隐私保护}

\begin{frame}{WSN~安全定位的解决方案}

未完成
\end{frame}

\section{位置信息应用}

\begin{frame}{WSN~位置信息在其它安全方案中的应用}
未完成
\end{frame}

\begin{frame}{参考文献}
\bibliographystyle{IEEEtran}
\bibliography{IEEEfull,wsn}
\end{frame}
\end{CJK*}
\end{document}
